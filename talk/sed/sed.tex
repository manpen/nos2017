\documentclass[aspectratio=169]{beamer}

\usepackage{color, colortbl}

\usepackage[USenglish]{babel}
\usepackage{amsmath,amsfonts,amssymb}
\usepackage{graphicx}
\usepackage{pdfpages}

\usepackage{latexsym}
\usepackage{xspace}

%TODO: Cyriax should install packages...
%\usepackage{siunitx}
%\sisetup{locale = DE}

\usepackage{xstring}
\usepackage{tikz}
\usepackage{pgf}


\usetikzlibrary{calc}   
\usetikzlibrary{positioning}
\usetikzlibrary{automata}

\usepackage{soul}

% Make it different to distinguish it from Manus stuff
\usetheme{Berlin}
\usecolortheme{dolphin}

\usepackage{cancel}
\usepackage{ulem}






\begin{document}
\title{
	Cyriax dem sein supertoller \texttt{sed} Vortrag
}
\author{
	Jonathan Cyriax Brast\\
	Manuel Penschuck \\
} 
\date{10. Juni 2017}

{
\setbeamertemplate{footline}{} 
\begin{frame}
  \titlepage
\end{frame}
}
\addtocounter{framenumber}{-1}

\newcommand{\bsl}{{\tt \textbackslash}}

%\frame{\frametitle{Outline}\tableofcontents}

%%%%%%%%%%%%%
\begin{frame}{\texttt{sed} - H\"a was?}
	{
	\Huge
	\only<1,3->{\texttt{sed}} % besteht aus den drei buchstaben
	\only<2>{\texttt{s e d}}\quad
	\only<4-5>{\only<5>{\xout}{''Suchen und Ersetzen in Dateien''}}
	\only<6->{''\textbf{S}tream \textbf{Ed}itor''}\\
	}
	\pause\pause\pause\pause\pause\pause
	{\tt sed} ist ein Linux Programm \\\pause
	{\tt sed} kann Text(-dateien) automatisch \"andern\\\pause
	{\tt sed} sucht Dinge und ersetzt sie durch Zeuch\\\pause
	{\tt sed} bekommt dazu Befehle wie es suchen und ersetzen soll\\\pause
	{\Large \quad\quad ergo folgt deshalb daraus:\\\pause}
	{\tt sed} ist viel cooler als Turing Maschinen\\\pause
	{\hspace{22em} $\square$ q.e.d. \texttt{\#isso}}
\end{frame}

\begin{frame}{{\tt sed} - Befehle} %obsolet, weil editwar. Aber bleibt drin, wegen realistischer
	Suchen und Ersetzen (alles andere is unwichtig):\\\pause
	{\tt s/     /      /}  - S wie ''suchen und ersetzen''\\\pause
	{\tt s/ foo / bar /}
		\pause\quad
		'' foo '' $\rightarrow$ '' bar ''\\\pause
	{\tt s/ .* / bar /}
		\pause\quad
		'' irgendwas '' $\rightarrow$ '' bar ''\\\pause
	{\tt s/ (foo) (bar) / \bsl2 \bsl1 /}
		\pause\quad
		'' foo bar '' $\rightarrow$ '' bar foo ''\\\pause
	Sonstige Befehle:\\\pause
	{\tt :irgendwo} Eine Sprungmarke\\\pause
	{\tt t irgendwo} Springe zur Sprungmarke\\\pause
	{\tt p } Drucke aus, was du grade denkst\\

\end{frame}


\begin{frame}{ {\tt sed} kann alle Teile von Manuels Turingmaschine bauen!}

	Manuels Maschine besteht aus:\\\pause
	Zust\"anden\\\pause
	Zunstands\"uberg\"angen\\\pause
	Band\\\pause
	Kopf\\
\end{frame}

\begin{frame}{ Zust\"ande und \"Uberg\"ange }
	Jeder Zustand hat einen Namen!\\\pause
	AlterZustand $\rightarrow$ NeuerZustand \only<2>{(wie ?)} % dummes bild!
	\\\pause
	{\tt s/AlterZustand/NeuerZustand/} \only<3>{(duh !)}
	\\\pause
	--separator--\\
	Aber der \"Ubergang h\"angt von einen Zeichen ab!\\\pause
	AlterZustand $0\rightarrow$ EinZustand \\ % dummes bild!
	AlterZustand $1\rightarrow$ AndererZustand \\ % dummes bild!
	\pause
	{\tt s/AlterZustand:0/EinZustand:0/}\\
	{\tt s/AlterZustand:1/NeuerZustand:1/}\\
\end{frame}

\begin{frame}{BandKopfApparat}
	%\begin{dummeIdee}
	@ sieht aus wie ein Kopf...\\\pause %Dumme Grafik!
	%\end{dummeIdee}
	Nach links gehen:\\
	''Hallo W@elt'' $\rightarrow$ ''Hallo @Welt''\\\pause
	\texttt{s/(.*)(.)@(.*)/\bsl1@\bsl2\bsl3/}\\\pause
	e lesen und \"a schreiben:\\
	''Hallo W@elt'' $\rightarrow$ ''Hallo W@\"alt''\\\pause
	\texttt{s/(.*)@e(.*)/\bsl1@\"a\bsl2/}\\\pause
	Beides Zusammen:\\
	''Hallo W@elt'' $\rightarrow$ ''Hallo @W\"alt''\\\pause
	\texttt{s/(.*)(.)@e(.*)/\bsl1@\bsl2\"a\bsl3/}\\
\end{frame}


\begin{frame}{Die {\tt sed} Turingmaschine}
	Wir brauchen:\\\pause
	1. Zustands\"ubergang \\\pause
	2. Lesen / Schreiben \\\pause
	3. Gehe nach (links/rechts)\\\pause
	Alles Gleichzeitig!\\\pause

	Gehe von ZUSTAND-ALT in ZUSTAND-NEU wenn du X liest und schreibe Y und\\
	\quad1. Gehe nach links:\\\pause
	\texttt{s/ZUSTAND-ALT:(.*)(.)@X(.*)/ZUSTAND-NEU:\bsl1@\bsl2Y\bsl3/}\\\pause
	\quad2. Gehe nach rechts:\\\pause
	\texttt{s/ZUSTAND-ALT:(.*)@X(.)(.*)/ZUSTAND-NEU:\bsl1Y@\bsl2\bsl3/}\\
\end{frame}



\end{document}

