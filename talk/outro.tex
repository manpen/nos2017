\begin{frame}{Warum sind Turing Maschinen \emph{wirklich} nützlich?}
	\textcolor{goetheblau}{Was haben wir gesehen?} \\
	$\left.\text{\parbox{0.5\textwidth}{
		\begin{itemize}
			\item<2-> \textcolor{goetheblau}{sed} kann eine \textcolor{emorot}{Turing Maschine} ausführen
			\item<3-> Eine \textcolor{emorot}{Turing Maschine} kann \textcolor{goetheblau}{sed} ausführen
			\item<4->[$\Rightarrow$] Beide sind gleich mächtig!
		\end{itemize}
	}}
	\hspace{1em}
	\only<-4>{\right.}
	\only<5->{\right\}\hspace{1em} \textbf{Turing Vollständig}}
	$
\end{frame}

\begin{frame}{Was können Turing Maschinen \textbf{nicht}?}
	\begin{itemize}
		\item Eingabe/Ausgabe beschreiben
		\pause
		\item Schnell rechnen
		\pause
		
		\vspace{1em}

		\item Für ein \textcolor{goetheblau}{beliebiges} Programm $P$ entscheiden, 
		\begin{itemize}
			\item \ldots ob $P$ fertig wird
			\pause
			\item \ldots ob $P$ für eine feste Eingabe, eine feste Ausgabe berechnet
			\pause
			\item \ldots ob $P$ eine nicht triviale Eigenschaft erfüllt
		\end{itemize}

		\vspace{1em}

		\item 
		Bei \textcolor{goetheblau}{beliebigen} \textquotedblleft Domino-Steinen\textquotedblright{} berechnen,\\ ob man oben und unten diesselben Worte legen kann
	\end{itemize}
\end{frame}


\begin{frame}{Was ist nicht Turing-Vollständig?}
	\begin{itemize}
		\item 
			Reine Daten
			\begin{itemize}
				\item (Werte-)Tabellen
				\item HTML, XML
				\item Rechenschieber
			\end{itemize}
		
			\vspace{1em}
			
		\pause
		
		\item Endliche Automaten
			\begin{itemize}
				\item 
					Turing Maschine mit schreibgeschüztem Band
					
				\item
					Turing Maschine, die den Kopf nur nach rechts bewegen darf
					
				\item
					Reguläre Sprachen
					
				\item
					\texttt{sed} ohne Sprünge
			\end{itemize}
		
		\pause
		
		\item \ldots
	\end{itemize}
\end{frame}

\setbeamertemplate{footline}{} 
\goethccBgTitel
\begin{frame}{}
	\titlepage
	\begin{tikzpicture}[overlay]
	\node[anchor=south east, xshift=-0.08\textwidth, at=(current page.south east)] {
		\parbox{0.3\textwidth}{
		\begin{center}
			\scalebox{1.5}{\qrcode{http://nos17.manuel.jetzt}}
			
			\vspace{1em}
			
			\small\textcolor{goetheblau}{\url{http://nos.manuel.jetzt}}
		\end{center}
	}};
	\end{tikzpicture}
\end{frame}
